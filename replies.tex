\documentclass [12pt,onecolumn,twoside,openright]{report}
\usepackage[utf8]{inputenc} \usepackage[T1]{fontenc} \usepackage[top=2.4cm,
bottom=2.4cm, left=2.4cm, right=2.4cm]{geometry} \usepackage{setspace}
\usepackage{color}

\begin{document} \begin{onehalfspacing} \begin{center} \large{Response to
Reviewer: Inferring food web structure from predator-prey body size
relationships} \end{center}

Dominique Gravel, Timoth\'ee Poisot, Camille Albouy, Laure Velez, David Mouillot
\begin{center} Reviewer 2: \end{center}

Overall, I think this is an excellent manuscript, nicely written, well explained
(despite some marginal points that might be clarified, see below) and addressing
an important topic. The overall quality of the description of the approach is
above average and addressing these points should be simple.

\medskip \textbf{\large{Main comments:}}

%---------------% 
\medskip \textbf{Comment}: The method assumes a general
relationship between the prey-size range and predator mass. While I think that
this assumption holds when predator and prey are pooled across broad size
classes, I think that many predator groups also have quite specific size ranges.
For instance, we have carried out laboratory studies (Brose et al. 2008, Rall et
al. 2011) with different predator groups on prey that vary in their sizes. Our
general finding were: (1) specific size ranges of prey are attacked, (2) the
lower and upper ends of theses size ranges can be mechanistically explained by
processes that constrain attack rates, and (3) these size ranges (centre and
width) vary across predator groups. The first two points strongly support the
approach presented here by Gravel et al., but the third indicates how
phylogenetic differences between predator groups could cause imprecise
predictions. Although I am a huge fan of this type of allometric scaling work, I
am sure (and I have experienced this quite often) that many other ecologists
would be very interested in an extended discussion where and how phylogenetic
differences among predators could be added to the model.

\medskip \textcolor{blue}{\textbf{Response:}} \textit{\textcolor{blue}{}}

%---------------% 
\medskip \textbf{Comment}: The three regression lines in Fig.
2 demonstrate the mean, maximum and minimum prey size of a predator. I found
this quite intriguing that for a given meta-community, this would predict how
the number of links to resources (the generality of the predator) and the number
of links to consumers (the vulnerability of a resource) depend on the body-mass
on the x-axis of the figure. We have analyzed these relationship for a variety
of food webs (Digel et al. 2011), but I could imagine that the methods
introduced here would allow a more thorough investigation of this relationship.
If possible, a general graph showing the relationship between generality and
predator body mass would be very interesting to see.

\medskip \textcolor{blue}{\textbf{Response:}} \textit{\textcolor{blue}{}}

%---------------% 
\medskip \textbf{Comment}: The decrease of the fraction of
potential links that are realized from local to landscape scales was the central
topic of one of our prior projects (Brose et al. 2004). You might find this
useful to support your assumption that meta-communities should include links
that are not realized at local scales – and meta food webs should thus contain
more links than the sum of the local food webs. This is exactly the pattern that
we documented for different types of ecosystems.

\medskip \textcolor{blue}{\textbf{Response:}} \textit{\textcolor{blue}{}}

%---------------% 
\medskip \textbf{Comment}: My apologies if these points rely
heavily on our own prior work. I found this manuscript very inspiring and also
highly related to our studies. Overall, I would recommend publication of this
manuscript, the minor points listed below should be no obstacle.

%---------------% 
\medskip Line 152: using the abbreviation B (often used for
biomass) for individual mass may be confusing – could be replaced by M.

\textcolor{blue}{\textbf{Response:}} \textit{\textcolor{blue}{}}

%---------------% 
\medskip Fig. 2: preys should be prey

\textcolor{blue}{\textbf{Response:}} \textit{\textcolor{blue}{}}

%---------------% 
\medskip Line 214: shouldn't it be up to eight dimensions?

\textcolor{blue}{\textbf{Response:}} \textit{\textcolor{blue}{}}

%---------------% 
\medskip Some references are incomplete (Eklöf, Mouquet,
Thuiller). 

\textcolor{blue}{\textbf{Response:}} \textit{\textcolor{blue}{}}

%----------------------------------%


\begin{center} Reviewer 3: \end{center}

This paper presents a simple method to infer trophic interactions in food-webs
from information on the body size of the species. While the topic is important
and such a method potentially very useful, I believe that the present work
suffers from many problems.

\medskip \textbf{\large{Main comments:}}

%---------------% 
\medskip \textbf{Comment 1}: It is not clear if the method is
useful to infer a particular food-web (as indicated in the title), or a
"metaweb". I was much confused about the scope of the method, which should be
more clearly stated. From the example with the Mediterranean fishes, it appears
that the proposed approach is useful to build a metaweb. It is not clear to me
if it could be of any use for the description of a "local" food-web. Also, even
at the metaweb level, it is apparent that additional information (spatial
co-occurrence) is needed to obtain a "sensible" metaweb. Ultimately, this
questions the usefulness of the method.

\medskip \textcolor{blue}{\textbf{Response:}} \textit{\textcolor{blue}{As stated
in the abstract, our method allows to ``infer the matrix of potential
interactions among a pool of species'', and does so regardless of whether one is
working at the local or regional scale. In the text, we give examples at both
these scales (locally in figures 2 and 3, regionnaly in figure 5). We do agree
with the fact that additional information will make the predictions better; this
is described at length in the text, and an example of this (including
co-occurence along the depth gradient) is pictured in Fig. 5b. This is also
explicitely written in the discussion (L. XXX-XXX).}}

%---------------% 
\medskip \textbf{Comment 2}: The authors illustrate the
approach with the use of a food-web based on fishes. It is known that body-size
is important to understand trophic interactions for fishes and for aquatic
systems in general. It is questionable that the approach would be useful for 1)
terrestrial food-webs, and 2) for "complete" food-webs encompassing a larger
array of taxa (this could be circumvented by applying different regressions for
different taxonomic groups; see the paper of Naisbit et al. 2011 in Ecology).

\medskip \textcolor{blue}{\textbf{Response:}} \textit{\textcolor{blue}{We agree
that the fit of the model is only as strong as the relationship between prey and
predator body-size. This is the purpose of Fig. 3; this relationship is also
described in section 3.1 ``Predictive performance''.}}

%---------------% 
\medskip \textbf{Comment 3}: Important contributions are not
discussed in the manuscript. To my knowledge, the first approach that was able
to predict trophic interactions was given by Ives and Godfray (Am. Nat. 2006),
but based on phylogenetic information (their method is admittedly much more
complex). The authors cite the paper of Petchey et al. (PNAS 2008), but it true
that the aim of their model was surely not to forecast possible interactions.
However, a quite similar approach was proposed by Rohr et al. (Am. Nat. 2010),
which can be easily used to predict interactions. Their first model (Body-size
model) is also very simple, based on the ratio of prey/predator body sizes, and
assumes an optimal ratio (if I am correct, it would be similar to the approach
of the authors with a slope fixed at a value of 1). Because of the simplicity of
the approach, a comparison of both methods could be useful (however, Rohr et al.
approach predicts a total number of interactions that is very close to the
number of interactions used in the "training" food-web; this could be simply
adjusted by decreasing the threshold (see their fig. 1) to a level corresponding
to the 5 and 95\% quantiles used by the authors).

\medskip \textcolor{blue}{\textbf{Response:}} \textit{\textcolor{blue}{}}

%---------------% 
\medskip \textbf{Comment 4}: The authors use Model 1
regression in their analysis. To me, this is problematic as measurement error on
body mass is similar for prey and predators. I recommend the use of Model 2
regression.

\medskip \textcolor{blue}{\textbf{Response:}} \textit{\textcolor{blue}{}}

%---------------% 
\medskip l.50: The use of "emergent" sounds strange here.
Please clarify.

\textcolor{blue}{\textbf{Response:}} \textit{\textcolor{blue}{}}

%---------------% 
\medskip 2nd paragraph: the metaweb concept is introduced
here, but it is not clear why. If the aim of the authors is to infer a metaweb,
this should be clearly stated here.

\textcolor{blue}{\textbf{Response:}} \textit{\textcolor{blue}{}}

%---------------% 
\medskip l.128: Here, it is apparent that the authors aim is
to infer a metaweb. Again, this should be clearly stated earlier (and even in
the title?)

\textcolor{blue}{\textbf{Response:}} \textit{\textcolor{blue}{}}

%---------------% 
\medskip l.139-140: How can you state this here without
performing an analysis? This is unclear to me (at least, this is not intuitive).


\textcolor{blue}{\textbf{Response:}} \textit{\textcolor{blue}{This is a basic
postulate of the niche model. We have added a reference to the original paper to
eliminate ambiguity.}}

%---------------% 
\medskip l.155: grey or dotted lines?

\textcolor{blue}{\textbf{Response:}} \textit{\textcolor{blue}{The
  \emph{dotted} lines. We corrected accordingly and apologize for the mistake.}}

%---------------% 
\medskip l.157: what are these betas?

\textcolor{blue}{\textbf{Response:}} \textit{\textcolor{blue}{}}

%---------------% 
\medskip l.167: should the $\alpha_1$ rather be a $\beta_1$
(see my former comment)? Please clarify.


\textcolor{blue}{\textbf{Response:}} \textit{\textcolor{blue}{No. The centroid
is a function of the two body sizes (hence $\alpha$), while the range limits are
a function of the predator body size and its range.}}

%---------------% 
\medskip l.199-201: if I understand correctly Fig. 3, the
increase in TSS is due mostly to the correct prediction of the fraction d
(predicted absences), and not to correctly predicted presences (fraction a),
which is likely the most interesting fraction for any practical reason. This
should be stated.

\textcolor{blue}{\textbf{Response:}} \textit{\textcolor{blue}{Whether the
correct prediction of absence or presence of an interaction is the most
interesting part of a model is ultimately a matter of personal taste, and
adequation to the study objectives. Given that (especialyl when no other traits
than body size are included), this method tends to over-estimate the number of
links, we argue that correctly predicting absences of links is an extremely
important feature.}}

%---------------% 
\medskip l.304: expected, not expect

\textcolor{blue}{\textbf{Response:}} \textit{\textcolor{blue}{We corrected
accordingly.}}

%---------------% 
\medskip l.511: I would not discuss the problem of ontogenic
shift in the figure legend.

\textcolor{blue}{\textbf{Response:}} \textit{\textcolor{blue}{We argue it is an
important piece of information, which was required by a previous reviewer. We
made no changes.}}

%---------------% 
\medskip l.515: Is their any justification behind the use of 5
and 95\% quantiles? Why not the usual 2.5 and 97.5\%, or 0 and 100\%?

\textcolor{blue}{\textbf{Response:}} \textit{\textcolor{blue}{}}

%---------------% 
\medskip l.522: black and open dots: I guess it is the
contrary.

\textcolor{blue}{\textbf{Response:}} \textit{\textcolor{blue}{}}

%---------------% 
\medskip l. 526: "with not pbservation" ?

\textcolor{blue}{\textbf{Response:}} \textit{\textcolor{blue}{We corrected
accordingly.}}

%---------------% 
\medskip Figure 2: An important difference between the
original and the parameterized niche model is apparent in this figure: here, the
$c_i$ and $r_i$ are constrained by the allometric regression, which is not the
case in the original model. From the distribution of the open dots, this appears
quite problematic in this figure (many open dots are outside the predicted
range; many predicted interactions are not existing), and this questions the
usefulness of the whole approach. Other traits may be needed; perhaps performing
several regressions for different broad taxonomic groups may also improve the
approach, and widen its applicability.

\textcolor{blue}{\textbf{Response:}} \textit{\textcolor{blue}{The fact that our
model over-predict the number of interactions is abundantly discussed in the
manuscript (along with why, and how to overcome it), so we will not elaborate
further on it in this reply. In the original niche model paper, it is true
that $c$ is not constrained by $n$. However, the purposes of the two models are
not the same. The niche model allows to generate a distribution of network
properties based only on connectance and richness. We add allometric
informations so as to be able to estimate species-level interactions in addition
to network properties. The allometric relationshipss on which we rely are well
documented. As for the fact that other traits can be added, this is discussed in
the manuscript at several places, including a whole figure. We made no changes
to the manuscript.}}

\end{onehalfspacing}

\end{document}
